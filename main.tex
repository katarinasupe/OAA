\documentclass[xcolor=dvipsnames, notheorems]{beamer}
\hypersetup{unicode}

\usepackage{algorithm2e}
\usepackage{adjustbox}
\usepackage[utf8]{inputenc}
\usepackage[croatian]{babel}
\usepackage{csquotes}
%\MakeOuterQuote{"}
\usepackage{tikz}
\usetheme{Boadilla}
\usecolortheme{lily}
\usepackage{amsthm}
\usepackage{graphicx}
\usepackage{tcolorbox}
\usepackage{listings}


\setbeamertemplate{theorems}[] % to number

\theoremstyle{plain} % insert bellow all blocks you want in italic
\newtheorem{theorem}{Teorem}[] % to number according to section

\theoremstyle{definition} % insert bellow all blocks you want in normal text
\newtheorem{definition}{Definicija}[] % to number according to section

\definecolor{Coral}{RGB}{255, 111, 97}
\definecolor{DarkBluishGray}{RGB}{47,52,53}
\setbeamercolor{title}{fg=Coral, bg=white}
\setbeamercolor{frametitle}{fg=DarkBluishGray, bg=Coral}
\setbeamercolor{title in head/foot}{fg=Coral, bg=DarkBluishGray}
\setbeamercolor{author in head/foot}{fg=Coral, bg=DarkBluishGray}
\setbeamercolor{date in head/foot}{fg=Coral, bg=DarkBluishGray}
\setbeamercolor{block title}{bg=Coral,fg=DarkBluishGray}

\setbeamertemplate{enumerate item}{%
  \usebeamercolor[bg]{item projected}%
  \raisebox{1.5pt}{\colorbox{Coral}{\color{DarkBluishGray}\footnotesize\insertenumlabel}}%
}
\usepackage{listings}

\usepackage[backend=bibtex,style=verbose-trad2]{biblatex}
\usepackage{biblatex}
\bibliography{literatura} 

\title{Binarni Euklidov algoritam}
\subtitle{Domaća zadaća - Oblikovanje i analiza algoritama}
\author{\texorpdfstring{Katarina Šupe}{Katarina Supe}}
\institute[PMF--MO]{Prirodoslovno-matematički fakultet --- Matematički odsjek}
\date{\today}

\begin{document}
\begin{frame}[plain]
\titlepage
\end{frame}
\begin{frame}{Zadatak}
Objasniti i implementirati efikasni Euklidov algoritam.\\
Program treba učitavati dva prirodna broja a i b za koje tražimo najveću zajedničku mjeru. Sam algoritam koristi njihov binarni zapis (jednostavno dijeljenje s dva - \emph{shift} i sl.).
\end{frame}
\begin{frame}{Najveća zajednička mjera}
\begin{definition}
Neka su $a$ i $b$ cijeli brojevi koji nisu oba nula. Za prirodan broj $d$ kažemo da je najveća zajednička mjera (ili najveći zajednički djelitelj) brojeva $a$ i $b$, i pišemo $d=NZM(a,b)$ , ako $d$ ima sljedeća svojstva:
\begin{enumerate}
    \item   $d\mid a$ i $d\mid b$
    \item  Za svaki prirodni $c$, $c\mid a$ i $c \mid b \Rightarrow c\mid d$
\end{enumerate}
\end{definition}
Kada su oba broja nula, svaki prirodni broj dijeli nulu, pa ne možemo primijeniti gornju definiciju. Stoga postavimo:
\[ NZM(0,0) = 0 \]
\end{frame}

\begin{frame}{Najveća zajednička mjera}
Iz definicije slijedi:\\
\[ NZM(a,b) = NZM(b,a) \]
\[ NZM(a,b) = NZM(-a,b) \]
\[ NZM(a, 0) = |a| \]
Još neka bitna svojstva:
\[ NZM(k \cdot a, k \cdot b) = k \cdot  NZM(a,b) \]
\[ \text{ako je } NZM(a,b) = 1 \text{ tada je } NZM(a, k \cdot b) =NZM(a, k) \]
\[ NZM(a, b) = NZM(a - b, b) \]
\end{frame}


\begin{frame}{Najveća zajednička mjera}
Iz osnovnog teorema aritmetike slijedi faktorizacija nekog prirodnog broja $a$ (do na poredak prostih faktora):
\[ a = 2^{a_2} \cdot 3^{a_3} \cdot 5^{a_5} \cdot 7^{a_7} \cdots =   \prod_{p \text{ prost}} p^{a_p} \]
gdje su $a_p$ jedinstveni nenegativni brojevi. Svi osim konačno mnogo eksponenata $a_p$ su jednaki nula.  \\
Po definiciji tada slijedi:
\[ NZM(a,b) = \prod_{p \text{ prost}} p^{\min(a_p, b_p)} \]
Npr. $a = 24 = 2^3 \cdot 3^1$, $b = 4 = 2^2 \cdot 3^0$ pa je $NZM(a,b) = 2^{\min(3,2)} \cdot 3^{\min(1,0)} = 2^2 \cdot 3^0 = 4$
\end{frame}

\begin{frame}
\frametitle{Općenito o Euklidovom algoritmu}
Gornji navedeni algoritam za pronalaženje najveće zajedničke mjere je jasan na papiru te vraća točan rezultat. Međutim, nije efikasan, jer zahtijeva faktorizaciju brojeva $a$ i $b$, što zahtijeva dijeljenje tih brojeva s prostim brojevima redom, sve dok $a$ i $b$ ne postanu $1$.\\
Euklid je u svojim \emph{Elementima} pronašao metodu za određivanje najveće zajedničke mjere dvaju cijelih brojeva, bez faktoriziranja danih brojeva.
\end{frame}

\begin{frame}
\frametitle{Općenito o Euklidovom algoritmu}
Odsad možemo promatrati nalaženje najveće zajedničke mjere dvaju nenegativnih brojeva, što je u skladu s našim zadatkom. Spomenimo dvije verzije Euklidovog algoritma za nalaženje najveće zajedičke mjere, koji su doveli do efikasnog binarnog algoritma. Prva verzija nalazi $NZM(a,b)$ oduzimanjem, a druga dijeljenjem s ostatkom.
\end{frame}
\begin{frame}{Euklidov algoritam s oduzimanjem}
Euklidov algoritam s oduzimanjem rekurzivno traži $NZM$ manjeg broja te razlike većeg i manjeg broja, sve dok ne postanu jednaki.
\lstinputlisting[language=c, basicstyle=\footnotesize]{euklid_oduzimanje.c} 
\end{frame}
\begin{frame}{Euklidov algoritam s dijeljenjem s ostatkom}
Dijelimo s ostatkom sve dok $a$ ne postane djeljiv s $b$. Tada je posljednji ostatak različit od $0$ upravo $NZM(a,b)$.
\lstinputlisting[language=c, basicstyle=\footnotesize]{euklid_dijeljenje.c} 
\end{frame}
\begin{frame}{Binarni algoritam}
Binarni GCD algoritam (\emph{greatest common divisor} = najveći zajednički djelitelj = najveća zajednička mjera) poznat je kao \emph{Stein}ov algoritam (\emph{Josef Stein}, 1967.) ili binarni Euklidov algoritam. Taj algoritam nalazi najveću zajedničku mjeru dvaju nenegativnih brojeva. \emph{Stein}ov algoritam koristi jednostavnije artimetičke operacije od običnog Euklidovog algoritma s dijeljenjem s ostatkom. Dijeljenje mijenja s aritmetičkim posmacima (\emph{shift}), usporedbama i oduzimanjem. 
\end{frame}
\begin{frame}{Ideja binarnog algoritma}
Pomoću svojstava najveće zajedničke mjere, dolazimo do algoritma:
\begin{enumerate}
\item Ako je $a$ jednak $b$, tada je $NZM(a,b) = a$.\\
\item Ako je $a$ jednak $0$, tada je $NZM(a,b) = b$. \\
\item Ako je $b$ jednak $0$, tada je $NZM(a,b) = a$.\\
\item Ako su $a$ i $b$ parni, tada je $NZM(a,b) = 2 NZM(a/2, b/2)$.\\
\item Ako je $a$ paran i $b$ neparan, tada je $NZM(a,b) = NZM(a/2, b)$.\\
\item Ako je $a$ neparan i $b$ paran, tada je $NZM(a,b) = NZM(a, b/2)$.\\
\item Ako su $a$ i $b$ neparni te $a > b$, tada je $a - b$ paran broj te je $NZM(a,b) = NZM((a-b)/2,b)$.\\
\item Ako su $a$ i $b$ neparni te $a < b$, tada je $b - a$ paran broj te je $NZM(a,b) = NZM((b-a)/2,a)$.
\end{enumerate}
\end{frame}

\begin{frame}{Primjer: $a = 49, b = 14$}
\only<1>{Računamo $NZM(49, 14)$}
\only<2>{Računamo $NZM(49, 7)$}
\only<3>{Računamo $NZM((49-7)/2, 7) = NZM(21,7)$}
\only<4>{Računamo $NZM((21-7)/2, 7) = NZM(7,7) = 7$}
\begin{enumerate}
\item \alert<4>{Ako je $a$ jednak $b$, tada je $NZM(a,b) = a$.}\\
\item Ako je $a$ jednak $0$, tada je $NZM(a,b) = b$. \\
\item Ako je $b$ jednak $0$, tada je $NZM(a,b) = a$.\\
\item Ako su $a$ i $b$ parni, tada je $NZM(a,b) = 2 NZM(a/2, b/2)$.\\
\item Ako je $a$ paran i $b$ neparan, tada je $NZM(a,b) = NZM(a/2, b).$\\
\item \alert<1>{Ako je $a$ neparan i $b$ paran, tada je $NZM(a,b) = NZM(a, b/2)$.}\\
\item \alert<2,3>{Ako su $a$ i $b$ neparni te $a > b$, tada je $a - b$ paran broj te je $NZM(a,b) = NZM((a-b)/2,b)$.}\\
\item Ako su $a$ i $b$ neparni te $a < b$, tada je $b - a$ paran broj te je $NZM(a,b) = NZM((b-a)/2,a)$.
\end{enumerate}
\end{frame}
\begin{frame}{Implementacija binarnog algoritma}
Ovaj algoritam naziva se \textbf{binarnim}, jer kao što smo već spomenuli, ne koristi obično dijeljenje brojeva, već samo dijeljenje s $2$. \\  
\textbf{Dijeljenje s 2}\\
Desni posmak (\emph{right shift}, \texttt{>>} u \texttt{C-u}):\\
Neka je $a$ nenegativni broj. Tada broj $a$ dijelimo s $2$ koristeći desnim posmakom za jedno mjesto, tj. $a\texttt{ >> } 1$.\\
\textbf{Provjera parnosti}\\
Bitovna konjunkcija (\texttt{AND}, \& u \texttt{C}-u):\\
Neka je $a$ paran nenegativni broj. Tada je $a$ \& $1$ jednako $0$.\\
Neka je $a$ neparan nenegativni broj. Tada je $a$ \& $1$ jednako $1$.\\
\end{frame}    

\begin{frame}{Implementacija binarnog algoritma}
Pokažimo, za primjer korištenja bitovne konjunkcije i desnog posmaka, kako bi se u \texttt{C}-u, u rekurzivnoj verziji algoritma, implementirale 5., 6. i 7. grana algoritma: 
\lstinputlisting[language=c, basicstyle=\footnotesize]{grane_binarni.c} 
\end{frame}
 \begin{frame}{Složenost}
 Algoritam zahtijeva $\mathcal{O}(n)$ koraka, gdje je $n$ broj bitova većeg od dva broja, kako svaka dva koraka reduciraju barem jedan od operanada za barem faktor 2. Svaki korak uključuje samo par aritmetičkih operacija ($\mathcal{O}(1)$). Ukoliko su brojevi veličine riječi, svaka artimetička operacija je jedna strojna operacija pa je broj strojnih operacija reda $\log(\max(a,b))$.\\
Asimptotska složenost ovog algoritma je $\boxed{\mathcal{O}(n^2)}$, kako aritmetičke operacije (oduzimanje i \emph{shift}) uzimaju linearno vrijeme za proizvoljno velike brojeve.  
 \end{frame}
 
\begin{frame}{Literatura}
  \printbibliography
  \nocite{*}
\end{frame}
\end{document}